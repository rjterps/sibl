\documentclass[border=10pt]{standalone}

\usepackage{tikz}
\usetikzlibrary{backgrounds}
\usepackage{xcolor}
\usepackage{pagecolor}
\definecolor{hred}{rgb}{1,0,0}
\definecolor{hivory}{rgb}{1,1,0.94}
\definecolor{hdarkblueblack}{RGB}{30,30,40}
\definecolor{hdarkgrayblack}{RGB}{35,35,35}
% \pagecolor{hdarkgrayblack}

% reference
% https://public.tableau.com/static/images/Ta/TableauColors/ColorPaletteswithRGBValues/1_rss.png
\definecolor{tabblue}{RGB}{31,119,180}
\definecolor{taborange}{RGB}{255,127,14}
\definecolor{tabgreen}{RGB}{44,160,44}
\definecolor{tabred}{RGB}{214,39,40}
\definecolor{tabpurple}{RGB}{148,103,189}
\definecolor{tabbrown}{RGB}{140,86,75}
\definecolor{tabpink}{RGB}{227,119,194}
\definecolor{tabgray}{RGB}{127,127,127}
\definecolor{tabolive}{RGB}{188,189,34}
\definecolor{tabcyan}{RGB}{23,190,207}

\begin{document}
\pagestyle{empty}

% set overall layout of tree
% \tikzstyle{level 0}=[red]  # doesn't work, so set in tree instead
\tikzstyle{level 1}=[taborange, level distance=2.0cm, sibling distance=2.5cm]
\tikzstyle{level 2}=[tabgreen, level distance=2.0cm, sibling distance=1.5cm]
\tikzstyle{level 3}=[tabred, level distance=2.0cm, sibling distance=1.5cm]

% define styles for bags and leaves
\tikzstyle{bag} = [text width=4em, text centered]
\tikzstyle{end} = [circle, minimum width=3pt, inner sep=0pt]

%\begin{tikzpicture}[grow=right, sloped]
\begin{tikzpicture}[sloped]
%\begin{tikzpicture}[grow=right]
\node[circle, draw, tabblue] {\tiny root}
	child {
		node[circle, draw] {\tiny \bf{00}}
			% child {
			% 	node[circle, draw] {\tiny \bf{0000}}
			% 	child {
			% 		node[circle, draw] {\tiny \bf{000000}}
			% 		edge from parent
			% 	}
			% 	child {
			% 		node[circle, draw] {\tiny \bf{000001}}
			% 		edge from parent
			% 	}
			% 	child {
			% 		node[circle, draw] {\tiny \bf{000010}}
			% 		edge from parent
			% 	}
			% 	child {
			% 		node[circle, draw] {\tiny \bf{000011}}
			% 		edge from parent
			% 	}
			% 	edge from parent
			% }
			% child {
			% 	node[circle, draw] {\tiny \bf{0001}}
			% 	edge from parent
			% }
			% child {
			% 	node[circle, draw] {\tiny \bf{0010}}
			% 	edge from parent
			% }
			% child {
			% 	node[circle, draw] {\tiny \bf{0011}}
			% 	edge from parent
			% }
		edge from parent
		% node[above] {\scriptsize $P(\overline{C})=0.99$}
	}
	child {
		node[circle, draw] {\tiny \bf{01}}
			% child {
			% 	node[circle, draw] {\tiny \bf{$\cdots$}}
			% 	edge from parent
			% }
		edge from parent
		% node[above] {\scriptsize $P(\overline{C})=0.99$}
	}
	child {
		node[circle, draw] {\tiny \bf{10}}
			% child {
			% 	node[circle, draw] {\tiny \bf{$\cdots$}}
			% 	edge from parent
			% }
		edge from parent
		% node[above] {\scriptsize $P(\overline{C})=0.99$}
	}
	child {
		node[circle, draw] {\tiny \bf{11}}
			child {
				node[circle, draw] {\tiny \bf{1100}}
				edge from parent
			}
			child {
				node[circle, draw] {\tiny \bf{1101}}
				edge from parent
			}
			child {
				node[circle, draw] {\tiny \bf{1110}}
				edge from parent
			}
			child {
				node[circle, draw] {\tiny \bf{1111}}
				child {
					node[circle, draw] {\tiny \bf{111100}}
					edge from parent
				}
				child {
					node[circle, draw] {\tiny \bf{111101}}
					edge from parent
				}
				child {
					node[circle, draw] {\tiny \bf{111110}}
					edge from parent
				}
				child {
					node[circle, draw] {\tiny \bf{111111}}
					edge from parent
				}
				edge from parent
			}
		edge from parent
	}; % remember trailing semi-colon after last child
\end{tikzpicture}

  % \begin{tikzpicture}
  %   [
  %     background rectangle/.style = {fill=hdarkgrayblack},
  %     show background rectangle
  %   ]

  %   \begin{scope}[shift={(3cm,-5cm)}, fill opacity=0.3,
  %     mytext/.style={text opacity=1,font=\large\bfseries}]

	 %  \draw[fill=green] (-3,0) circle (2) node[hivory] (domain) {domain $\mathcal{A}$};
  % 	  \draw[fill=blue] (3,0) circle (3) node[hivory] {\hspace{4em} co-domain $\mathcal{B}$};
  %     \draw[fill=red] (1.5,0) circle (1) node[hivory] (image) {image $B$};

  %     \draw[->,hivory] (domain) edge[bend right] node [below] {$f$} (image);

  %   \end{scope}
  % \end{tikzpicture}

\end{document}
