\documentclass[11pt, oneside]{article}   	% use "amsart" instead of "article" for AMSLaTeX format
\usepackage{geometry}                		% See geometry.pdf to learn the layout options. There are lots.
\geometry{letterpaper}                   		% ... or a4paper or a5paper or ... 
%\geometry{landscape}                		% Activate for rotated page geometry
%\usepackage[parfill]{parskip}    		% Activate to begin paragraphs with an empty line rather than an indent
\usepackage{graphicx}				% Use pdf, png, jpg, or eps§ with pdflatex; use eps in DVI mode
								% TeX will automatically convert eps --> pdf in pdflatex		
\usepackage{amssymb}

% % listings begin
\usepackage{listings}
% \providecommand{\GitRemote}{}
% \providecommand{\GitIdentifier}{master}
% \providecommand{\GitCheckout}[2][\GitIdentifier]{%
% % #1 being the version/branch
% % #2 being the file
% | \string"git archive --remote=\GitRemote #1 \detokenize{#2} 2>/dev/null | tar --extract --file - --to-stdout \string"%
% }
% % listings end

\usepackage{tikz}
\usetikzlibrary{positioning}
\usepackage{mathtools}

%SetFonts

%SetFonts

% External dependencies
\usepackage{pagecolor}
\input{../../../../include/hmath_descendant.tex}
% \input{https://github.com/hovey/include/blob/a9bf6db0394b5ba19e69653519a96c10eeb1e133/hmath_descendant.tex}
% https://tex.stackexchange.com/questions/500463/using-listinputlisting-to-include-a-specific-git-commit
% https://git-scm.com/docs/git-archive
% https://tex.stackexchange.com/questions/500463/using-listinputlisting-to-include-a-specific-git-commit

\title{Mesh Smoothing}
\author{C.B.~Hovey}
%\date{}							% Activate to display a given date or n date

\begin{document}
\maketitle
%\section{}
%\subsection{}

% renewcommand{\GitRemote}{ssh://git@trac.sagemath.org/sage.git}
% \lstinputlisting{\GitCheckout{src/sage/coding/goppa.py}}
% \renewcommand{\GitRemote}{ssh://git@github.com:hovey/include.git}
% \lstinputlisting{\GitCheckout{hovey/include/hmath_descendant.tex}}

% \lstinputlisting{|\string"git archive --remote=ssh://git@server/repo.git VERSION path/to/file 2>/dev/null | tar --extract --file - --to-stdout\string"}
% \lstinputlisting{|\string"git archive --remote=ssh://git@github.com:hovey/include.git v0.0.1 hmath_descendant.tex 2>/dev/null | tar --extract --file hmath_descendant.tex --to-stdout\string"}
% \lstinputlisting{|\string"git archive --remote=https://git@github.com/hovey/include.git include/hmath_descendant.tex 2>/dev/null | tar --extract --file hmath_descendant.tex --to-stdout\string"}
% \lstinputlisting{|\string"git archive --remote=https://git@github.com/hovey/include.git v0.0.1 include/hmath_descendant.tex 2>/dev/null | tar --extract --file - --to-stdout\string"}


Let the subject point $\vp \in \realnsd$ have coordinates $(x)$ in 1D, 
$(x, y)$ in 2D, and $(x, y, z)$ in 3D.  
The subject point connects to $n$ neighbor points
$\vq_i$ for $i \in [1, n]$ though $n$ edges.  
In Fig.~\ref{fig:node_p}, for example, the point $\vp$ connects for four 
neighbors.

\begin{figure}[htb]
  \begin{center}

    \begin{tikzpicture}
      \draw[lightgray, thin] (0,0) -- (1,0);  % x-axis
      \draw[lightgray, thin] (0,0) -- (0,1); % y-axis
      \draw[green, thick] [-stealth](0,0) -- (4,4);  % center, pbar
      \draw[blue, thick] [-stealth](0,0) -- (2.5,6);  % p
      \draw[dashed, red, thick] [-stealth](4,4) -- (2.5,6);  % g

      \draw[dotted, blue, thick] (4,8) -- (2.5,6);  % p to north
      \draw[dotted, blue, thick] (0,4) -- (2.5,6);  % p to west
      \draw[dotted, blue, thick] (4,0) -- (2.5,6);  % p to south
      \draw[dotted, blue, thick] (8,4) -- (2.5,6);  % p to east

      \filldraw[black] (0,0) circle (2pt) node[anchor=east]{$O$};
      \filldraw[black] (4,4) circle (2pt) node[anchor=south]{$\bar{\vp}$};
      \filldraw[black] (2.5,6) circle (2pt) node[anchor=south]{$\vp$};
      \filldraw[black] (3.25,5) circle (0pt) node[anchor=south]{$\vg$};
      \filldraw[black] (4,8) circle (2pt) node[anchor=south]{$\vq_i$};
      \filldraw[black] (0,4) circle (2pt) node[anchor=south]{$\vq_{i+1}$};
      \filldraw[black] (4,0) circle (2pt) node[anchor=south]{$\vq_{n-1}$};
      \filldraw[black] (8,4) circle (2pt) node[anchor=south]{$\vq_n$};
    \end{tikzpicture}

  \end{center}

  \caption{Subject node $\vp$ with edge connections (dotted lines) 
  to neighbor nodes $\vq_i$ 
  with $i \in [1, n]$ (without loss of generality, 
  the specific example of $n = 4$ is shown).  
  The average position of all neighbors of $\vp$ 
  is denoted $\bar{\vp}$, and the gap (dashed line) 
  from $\bar{\vp}$ to $\vp$ is gap $\vg$.}
\label{fig:node_p} % label must come after caption 
\end{figure}
Let $\bar{\vp}$ denote the average position of all neighbors of $\vp$ and be 
defined as
\be
  \bar{\vp} \coloneqq \frac{1}{n} \sum_{i=1}^n \vq_i.
\ee
Let the gap vector $\vg$ be defined as originating at $\bar{\vp}$ and terminating
at $\vp$, such that
\be
 \vg \coloneqq \vp - \bar{\vp}, \;\;\; \mbox{since} \;\;\; \bar{\vp} + \vg = \vp.
\ee
Let $\lambda \in \realplus \subset (0, 1)$ be a scaling factor for the 
gap $\vg$.  Then we seek to iteratively update the position of $\vp^k$ at the
$k^{\mbox{\tiny th}}$ iteration by an amount $\lambda \vg^k$ to $\vp^{k+1}$ as
\begin{align}
  \vp^{k+1} & \coloneqq \vp^k - \lambda \vg^k, \hspace{0.5cm} \mbox{since} \\
  \bar{\vp} & = \vp - \vg \hspace{0.5cm} \mbox{when } \lambda = 1.
\end{align}
We typically select $\lambda < 1$ to avoid overshoot of the update.  Following
are two iterations for $\lambda = 0.1$ and initial positions $\vp = 1.5$ and
$\bar{\vp} = 0.5$ (given two neighbors, one at 0.0 and one at 1.0, that 
never move), a simple 1D example:

\begin{table}[htb]
  \caption{Two iteration update of a 1D example.}
  \label{tab:update_example} % label must come after caption 
  \centering
  \begin{tabular}{c|c|c|l|l}
    $k$ & $\bar{\vp}$ & $\vp^k$ & $\vg^k = \vp^k - \bar{\vp}$ & $\lambda \vg^k$ \\
   \hline
   \hline
   0 & 0.5 & 1.5 & 1.0 & 0.1 \\
   1 & 0.5 & 1.5 - 0.1 = 1.4 & 0.9 & 0.09 \\
   2 & 0.5 & 1.4 - 0.09 = 1.31 & 0.81 & 0.081
  \end{tabular}
\end{table}


% \begin{figure}[htb]
%   \begin{center}
% 
%     \begin{tikzpicture}[
%     roundnode/.style={circle, draw=green!60, fill=green!5, very thick, minimum size=7mm},
%     squarednode/.style={rectangle, draw=red!60, fill=red!5, very thick, minimum size=5mm},
%     ]
%     %Nodes
%     \node[roundnode] (center)                  {$\vp$};
%     \node[roundnode] (upper) [above=of center] {$\vq_1$};
%     \node[roundnode] (left)  [left=of center]  {$\vq_2$};
%     \node[roundnode] (lower) [below=of center] {$\vq_i$};
%     \node[roundnode] (right) [right=of center] {$\vq_{n}$};
%     
%     %Lines
%     \draw[-] (center.north) -- (upper.south);
%     \draw[-] (center.west) -- (left.east);
%     \draw[-] (center.south) -- (lower.north);
%     \draw[-] (center.east) -- (right.west);
%     % \draw[-] (upper.south) -- (center.north);
%     % \draw[-] (right.south) .. controls +(down:7mm) and +(right:7mm) .. (lower.east);
%     \end{tikzpicture}
% 
%   \end{center}
% 
%   \caption{Subject node $\vp$ with edge connections to neighbor nodes $\vq_i$ with $i \in [1, n]$.}
% \label{fig:single_node} % label must come after caption 
% \end{figure}


\end{document}

